% !TEX TS-program = pdflatex
% !TEX encoding = UTF-8 Unicode

% This file is a template using the "beamer" package to create slides for a talk or presentation

\documentclass{beamer}


\mode<presentation>
{
  \usetheme{default}
  % or ...

%  \setbeamercovered{transparent}
  % or whatever (possibly just delete it)
}


\usepackage[english]{babel}

\usepackage[utf8]{inputenc}

\usepackage{times}
\usepackage[T1]{fontenc}
% Or whatever. Note that the encoding and the font should match. If T1
% does not look nice, try deleting the line with the fontenc.



\title[Building a database on S3] % (optional, use only with long paper titles)
{Presentation of ``Building a database on S3''}

%\subtitle
%{for the seminar in TDT4150 ``Avanserte Databasesystemer''}

\author[Odd M. Trondrud]{Odd M. Trondrud}

\institute[Norwegian University of Technology and Science] 
{
  Presentation of seminar article\\
  TDT4150 ``Avanserte Databasesystemer''\\
  autumn semester of 2013\\
  NTNU
%  No department\\
%  I'm just a student
}

\date[]
{2013-10-22}



% If you have a file called "university-logo-filename.xxx", where xxx
% is a graphic format that can be processed by latex or pdflatex,
% resp., then you can add a logo as follows:

% \pgfdeclareimage[height=0.5cm]{university-logo}{university-logo-filename}
% \logo{\pgfuseimage{university-logo}}



% Delete this, if you do not want the table of contents to pop up at
% the beginning of each subsection:
\AtBeginSubsection[]
{
  \begin{frame}<beamer>{Outline}
    \tableofcontents[currentsection,currentsubsection]
  \end{frame}
}


% If you wish to uncover everything in a step-wise fashion, uncomment
% the following command: 

%\beamerdefaultoverlayspecification{<+->}


\begin{document}

\begin{frame}
  \titlepage
\end{frame}

\begin{frame}{Outline}
  \tableofcontents
  % You might wish to add the option [pausesections]
\end{frame}



\section{Motivation}

\subsection{The Basic Problem That We Studied}

\begin{frame}{Make Titles Informative. Use Uppercase Letters.}{Subtitles are optional.}
  % - A title should summarize the slide in an understandable fashion
  %   for anyone how does not follow everything on the slide itself.

  \begin{itemize}
  \item
    Use \texttt{itemize} a lot.
  \item
    Use very short sentences or short phrases.
  \end{itemize}
\end{frame}

\begin{frame}{Make Titles Informative.}

  You can create overlays\dots
  \begin{itemize}
  \item using the \texttt{pause} command:
    \begin{itemize}
    \item
      First item.
      \pause
    \item    
      Second item.
    \end{itemize}
  \item
    using overlay specifications:
    \begin{itemize}
    \item<3->
      First item.
    \item<4->
      Second item.
    \end{itemize}
  \item
    using the general \texttt{uncover} command:
    \begin{itemize}
      \uncover<5->{\item
        First item.}
      \uncover<6->{\item
        Second item.}
    \end{itemize}
  \end{itemize}
\end{frame}


\subsection{Previous Work}

\begin{frame}{Make Titles Informative.}
\end{frame}

\begin{frame}{Make Titles Informative.}
\end{frame}



\section{Our Results/Contribution}

\subsection{Main Results}

\begin{frame}{Make Titles Informative.}
\end{frame}

\begin{frame}{Make Titles Informative.}
\end{frame}

\begin{frame}{Make Titles Informative.}
\end{frame}


\subsection{Basic Ideas for Proofs/Implementation}

\begin{frame}{Make Titles Informative.}
\end{frame}

\begin{frame}{Make Titles Informative.}
\end{frame}

\begin{frame}{Make Titles Informative.}
\end{frame}



\section*{Summary}

\begin{frame}{Summary}

  % Keep the summary *very short*.
  \begin{itemize}
  \item
    The \alert{first main message} of your talk in one or two lines.
  \item
    The \alert{second main message} of your talk in one or two lines.
  \item
    Perhaps a \alert{third message}, but not more than that.
  \end{itemize}
  
  % The following outlook is optional.
  \vskip0pt plus.5fill
  \begin{itemize}
  \item
    Outlook
    \begin{itemize}
    \item
      Something you haven't solved.
    \item
      Something else you haven't solved.
    \end{itemize}
  \end{itemize}
\end{frame}



% All of the following is optional and typically not needed. 
\appendix
\section<presentation>*{\appendixname}
\subsection<presentation>*{For Further Reading}

\begin{frame}[allowframebreaks]
  \frametitle<presentation>{For Further Reading}
    
  \begin{thebibliography}{10}
    
  \beamertemplatebookbibitems
  % Start with overview books.

  \bibitem{Author1990}
    A.~Author.
    \newblock {\em Handbook of Everything}.
    \newblock Some Press, 1990.
 
    
  \beamertemplatearticlebibitems
  % Followed by interesting articles. Keep the list short. 

  \bibitem{Someone2000}
    S.~Someone.
    \newblock On this and that.
    \newblock {\em Journal of This and That}, 2(1):50--100,
    2000.
  \end{thebibliography}
\end{frame}

\end{document}


