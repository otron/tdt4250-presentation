% !TEX TS-program = pdflatex
% !TEX encoding = UTF-8 Unicode

\documentclass[11pt]{report} 

\usepackage[utf8]{inputenc}

%%% PAGE DIMENSIONS
\usepackage{geometry}
\geometry{a4paper} 
\geometry{margin=2in}
%   read geometry.pdf for detailed page layout information

\usepackage{graphicx} % support the \includegraphics command and options

% \usepackage[parfill]{parskip} % Activate to begin paragraphs with an empty line rather than an indent

%%% PACKAGES
\usepackage{booktabs} % for much better looking tables
\usepackage{array} % for better arrays (eg matrices) in maths
\usepackage{paralist} % very flexible & customisable lists (eg. enumerate/itemize, etc.)
\usepackage{verbatim} % adds environment for commenting out blocks of text & for better verbatim
\usepackage{subfig} % make it possible to include more than one captioned figure/table in a single float


%%% HEADERS & FOOTERS
\usepackage{fancyhdr} 
\pagestyle{fancy} 
\renewcommand{\headrulewidth}{0pt} 
\lhead{}\chead{}\rhead{}
\lfoot{}\cfoot{\thepage}\rfoot{}

%%% SECTION TITLE APPEARANCE
\usepackage{sectsty}
% \allsectionsfont{\sffamily\mdseries\upshape} % (See the fntguide.pdf for font help)
% (This matches ConTeXt defaults)

%%% ToC (table of contents) APPEARANCE
\usepackage[nottoc,notlof,notlot]{tocbibind} % Put the bibliography in the ToC
\usepackage[titles,subfigure]{tocloft} % Alter the style of the Table of Contents
\renewcommand{\cftsecfont}{\rmfamily\mdseries\upshape}
\renewcommand{\cftsecpagefont}{\rmfamily\mdseries\upshape} % No bold!



\title{Brief Article}
\author{The Author}
%\date{} % Activate to display a given date or no date (if empty),
         % otherwise the current date is printed 

\begin{document}
\maketitle

\section{First section}

And so it begins: with nothing.
Or an amount of something that is indiscernable from nothing.
You blink.
There's definitely something now.
A whole bunch of something.
Where'd it all come from?

The article, \%TITLE, lays down an outline of how one could build a near infinitely scalable database system on top of Amazon's Simple Storage Service (S3).
The proposed system exploits the high level of availability Amazon offers with its S3\footnote{$99.9\%$ uptime as of November 2013}




\section{Security}
The article barely touches on security concerns related to such a truly distributed system which it proposes.
It states that read and/or write privileges to a collection (i.e. a \textit{bucket} in S3 terminology) can be restricted on a per-client basis by the client that owns the collection, and that a security infrastructure \textit{can} be implemented on top of S3 however it does not present an explanation of \textit{how}.
This is somewhat understandable as the definition of ``client'' provided by the article\footnote{"\[\dots\] refer to software artifacts that retrieve pages from S3 and write pages back to S3."} is rather vague and implementing a proper security infrastructure for a system one knows very little about can be difficult.

However, from examining what security features Amazon now\footnote{November 2013} offers along with its SQS and S3 services it is possible to deduce with what kind of system a certain level of security is possible to achieve.

\subsection{Data Encryption}
Amazon offers Server-Side Encryption of data stored on S3.
However one should not trust others to keep one's own secrets.
Wholly relying on the offered Server-Side Encryption means that data arrives at Amazon's data centres unencrypted.
Regardless of how much trust one places in the service provider, this approach renders the system vulnerable to a man in the middle attack. 

Client-Side Encryption is possible as objects in S3 are simply byte streams with some S3 metadata (such as the object's URI).
Performing Client-Side Encryption increases the complexity of the task of obtaining the data.
However it also increases the complexity of the system as each client would have to be able to decrypt the data.


would require that an attacker is also in posession of the private key used to encrypt the data 

S3 a
If unencrypted data travels thr
is at some point outside of a controlled closed system 
If data arrives at Amazon's centres unencrypted 
However, if your data is unencrypted as it arrives Amazon's 

Client-Side Encryption would require that all involved clients 

\subsection{Traffic Encryption}


\section{Eventual Consistency}
Messages can be retained in SQS for up to 14 days.

\section{How bad is the eventual consistency?}
We don't know! The article didn't test it.

\subsection{A subsection}

More text.

\end{document}
